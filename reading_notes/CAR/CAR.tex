\documentclass[a4paper,12pt]{article}
\usepackage[english]{babel}
\usepackage[utf8]{inputenc}

%
% For alternative styles, see the biblatex manual:
% http://mirrors.ctan.org/macros/latex/contrib/biblatex/doc/biblatex.pdf
%
% The 'verbose' family of styles produces full citations in footnotes, 
% with and a variety of options for ibidem abbreviations.
%
\usepackage{graphicx}
\usepackage{csquotes}
\usepackage[style=verbose-ibid,backend=bibtex]{biblatex}
\bibliography{sample}

\usepackage{lipsum} % for dummy text

\title{Complex Answer Retrieval \\{\large Reading Notes}}
\author{Shayan Amani}

\date{\today}

\begin{document}
\maketitle

According to the two mentioned papers, we clearly get a whole picture of how the mentioned TREC's track progressed in a span of two years. Based on pragmatic examples, the authors have tried to illustrate different CAR tasks. Compared to the previous year, an improvement in data set has been reported from volume and also quality point of view so they went through a more sophisticated process of data set preparation. As a measure of evaluation a dual process of assessment of submissions to CAR track has proposed which comprised of the automatic and the manual ground truth signals. Due to a major alteration in paragraphs available in Wiki-18 in a comparison with Wiki-16, the former automatic assessment procedure utilized in Y1 has been rendered obsolete. By taking a side-by-side collation view of the variety of submission's results, we could noticeably observe that the overall result has improved. This improvement happened in terms of measures such as RPrec, NDCG, MAP, etc. in both declared tasks. 

\end{document}